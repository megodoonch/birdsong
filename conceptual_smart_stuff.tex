\documentclass[12pt]{article}
\usepackage{mathptmx}% http://ctan.org/pkg/mathptmx
\usepackage{times}
\usepackage{amsmath}
\usepackage{amsthm}
 \usepackage{amsfonts}
\theoremstyle{definition}
\newtheorem{definition}{Definition}[section]


\title{People who want to parse bigrams and finite state machines\\good}
\author{Meaghan ``geitje'' Fowlie and Floris ``konijntje'' van Vugt}


\begin{document}

\maketitle

\section{Definitions}

%\newcommand\STATES{\mathcal{S}}
\newcommand\STATES{\mathbb{S}}
%\newcommand\OPS{\mathcal{O}}
\newcommand\OPS{\mathbb{O}}
%\newcommand\BIGR{\mathcal{B}}
\newcommand\BIGR{\mathbb{B}}
\newcommand\FSA{\textsc{FSA}}
%\newcommand\PARSES{\mathcal{P}}
\newcommand\PARSES{\mathbb{P}}
\newcommand\SC{\textsc{SC}}
\newcommand\TC{\textsc{TC}}
\newcommand\N{\mathbb{N}}

The set of all possible states is $\STATES$ and the set of all operations is $\OPS$ and the set of all bigrams is $\BIGR$



\begin{definition}[finite state machines]
  A finite state automaton (Meaghan TODO) $\FSA$
  Set of tuples $(q_i,e,q_j)$ where $q_i,q_j\in\STATES$ and $e\in\OPS$ meaning that you can transition from state $q_i$ to $q_j$ by emitting $e$.
\end{definition}

\begin{definition}[transition probabilities]
  A probability assignment $\phi$ is a function from elements of the finite state automaton (i.e. transitions) to probabilities, such that the sum of all probabilities with the same left hand side is 1 (TODO: make that precise).
\end{definition}


\begin{definition}[route]
  A \emph{route} is a route through the $\FSA$ of say $n$ steps, defined as a tuple $(Q,E)$ where $Q$ is the sequence of states visited, i.e.
  $Q=(q_i\in\STATES|i<n)$, and $E$ is the sequence of emissions, i.e. $E=(e_i\in\OPS|i<n-1)$.
\end{definition}


\begin{definition}[parse]
  A \emph{parse} is what Meaghan's fanatic function currently outputs, more or less.
  Given a sentence $s$ we define a set $\PARSES(s) = \{ (b,r) \}$ where each $b$ is a sequence of alphabet elements (bigrams), and $r$ is a route through the operations $\FSA$.
\end{definition}


\begin{definition}[state counts]
  Given a route $(Q,E)$, we define $\SC_Q : \STATES \rightarrow\N$ as follows: $\SC_Q(q) = \#\{ q'\in \STATES | q'=q \}$
\end{definition}


\begin{definition}[transition counts]
  Given a route $(Q,E)$, we define $\TC_{Q,E} : \FSA \rightarrow\N$ as follows: $\TC_{Q,E}(q,e,q') = \#\{ i<n | Q(i)=q,E(i)=e,Q({i+1})=q'\}$
\end{definition}



We want to estimate the likelihood of each rule in a given corpus. For this, we iteratively re-estimate the rule probabilities: given a rule probability assignment, we can re-estimate the probabilities of every parse of a sentence, and then this changes the probabilities that each rule was used.

Roughly speaking, a transition from state $q$ to $q'$ can be estimated to happen with the following likelihood:
$$\phi(q,q') = \frac{\textrm{expected number of times we went }q\rightarrow q'}{\textrm{expected number of times we were in }q}$$

This leads to the following definition:

\begin{definition}[sentence-level update rule]
  Given a sentence $s$ and given a probability assignment $\phi$ we can define an updated probability assignment $\phi'$ as follows:

  $$\phi_s'(q,e,q') = \frac{\sum_{(b,r)\in\PARSES(s)}p_\phi(b,r)~\TC_r(q,e,q')}{\sum_{(b,r)\in\PARSES(s)}p_\phi(b,r)~\SC_r(q)}$$
\end{definition}

%\begin{definition}[sentence-level update rule]
%  Given a sentence $s$ and given a probability assignment $\phi$ we can define an updated probability assignment $\phi'$ as follows:%
%
%  $$\phi_s'(q,e,q') = \sum_{(b,r)\in\PARSES(s)}\frac{p_\phi(b,r)}{\sum_{(b',r')\in\PARSES(s)}p_\phi(b',r')}\frac{\TC_r(q,e,q')}{\SC_r(q)}$$
%\end{definition}

%We can write more simply $p_\phi(s)=\sum_{(b',r')\in\PARSES(s)}p_\phi(b',r')$ so that

%$$\phi_s'(q,e,q') = \frac{1}{p(s)}\sum_{(b,r)\in\PARSES(s)}p(b,r)\frac{\TC_r(q,e,q')}{\SC_r(q)}$$


Similarly, given a corpus, we compute the updates based on all parses in parallel:
\begin{definition}[corpus update rule]
  Given a corpus $C$ and given a probability assignment $\phi$ we can define an updated probability assignment $\phi'$ as follows:

 $$\phi_C'(q,e,q') = \frac{\sum_{s\in C}\sum_{(b,r)\in\PARSES(s)} p_\phi(b,r)~ \TC_r(q,e,q')}{\sum_{s\in C}\sum_{(b,r)\in\PARSES(s)} p_\phi(b,r)~ \SC_r(q)}$$ 
\end{definition}

%Floris' version:
%$$\phi_C'(q,e,q') = \frac{\sum_{s\in C}p(s)\phi_s'(q,e,q')}{\sum_{s\in C}p(s)}$$ 


The sums can't be combined into one sum because sometimes there are parses that visit state $q$ but don't follow transition $(q,e,q')$. However, we still want that visit in the state counts for that transition.

We can't use the probability of the whole corpus because not all sentences have parses that visit all states.



We leave it as an exercise to the reader to show that this equivalent to summing over all parses.


TODO:
\begin{itemize}
\item Prove correctness (rule probabilities with the same left hand side sum to one)
\item Prove convergence
\end{itemize}


\end{document}

%%% Local Variables:
%%% mode: latex
%%% TeX-master: t
%%% End:
